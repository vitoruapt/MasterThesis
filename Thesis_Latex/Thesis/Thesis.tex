\documentclass[12pt,a4paper,twoside,english]{book}
\usepackage[]{babel}
\usepackage[T1]{fontenc}
\usepackage[latin1]{inputenc}
\usepackage{siunitx}
%\usepackage[italian]{babel}
\usepackage{DTGtesi}
%\usepackage{latexsym}
\usepackage{booktabs}
% Pacchetto per l'inserimento del codice Matlab
\usepackage{mcode}
\lstloadlanguages{MATLAB}
\usepackage{xcolor}
%\usepackage{listings}
%\lstset{language=Matlab}
% Pacchetto per l'inserimento delle formule matematiche
%\usepackage{amsmath}
\usepackage{etoolbox}
\AtBeginEnvironment{quote}{\singlespacing\small}
\usepackage{amsmath,amssymb}
\newcommand{\numberset}{\mathbb}
\newcommand{\C}{\numberset{C}}
\newcommand{\R}{\numberset{R}}
%\usepackage{subfigure}
\usepackage{caption,subcaption}
\usepackage{algorithm}
\usepackage{algpseudocode}
\usepackage[font={small,it}]{caption}
\addto\captionsitalian{
	\renewcommand{\contentsname}
	{Contents} % ToC will show "Index" instead of "Content"
}

\graphicspath{{./figure/}}

\renewcommand{\vec}[1]{\ensuremath{\boldsymbol{\mathit{#1}}}}

\hypersetup{
  % pdfpagelayout=SinglePage, % default
  % pdfpagemode=UseOutlines,  % default
  % bookmarksopen,            % default
  % bookmarksopenlevel=2,     % default
  linkcolor=black,
  pdftitle=Title,
  pdfauthor=Alberto Franco,
  pdfsubject=Master Thesis in Automation Engineering,
  pdfkeywords=Master Thesis Automation}
  % Queste informazioni non vengono stampate, ma sono conservate nel documento pdf. Sono consultabili col menu "File>Document Properites>Description". Vengono buone a scopi archivistici.

%%%%%%%%%%%%%%%%%%%%%%%%%%%%%%%%%%%%%%%%%%%%%%%%%%%%%%%
%          Numerazione delle formule                  %
% Se non specificato altrimenti, il LaTeX numera le   %
% formule come (capitolo.formula) (per esempio (2.5)  %
% e` la quinta formula del secondo capitolo).         %
% Con le istruzioni seguenti invece la numerazione    %
% diventa (capitolo.sezione.formula) (per esempio     %
% (3.2.6) e` la sesta formula della seconda sezione   %
% del terzo capitolo):                                %
%%%%%%%%%%%%%%%%%%%%%%%%%%%%%%%%%%%%%%%%%%%%%%%%%%%%%%%



%%%%%%%%%%%%%%%%%%%%%%%%%%%%%%%%%%%%%%%
% Dati per la prima pagina della tesi %
%%%%%%%%%%%%%%%%%%%%%%%%%%%%%%%%%%%%%%%

\definecolor{SchoolColor}{rgb}{0.71, 0, 0.106} % UNIPD red

\school{Engineering}
\department{Information Engineering}
\corsodilaureamagistrale{Ingegneria dell'Automazione}
\titolo{\textcolor{SchoolColor}{\bfseries Local Path Planning with Moving}\\\textcolor{SchoolColor}{\bfseries Obstacle Avoidance based on}\\\textcolor{SchoolColor}{\bfseries  Adaptive MPC in ATLASCAR2}}
\supervisor[Prof.]{Angelo Cenedese}
\cosupervisor[Prof.]{Vitor Santos}
\mastercandidate{Alberto Franco}
\matricola{1156523}
\date{15$^{\text{th}}$ April 2019}
\annoaccademico{2018/2019}
\dedica{Povera mente,\\io ti uccido ogni giorno con le mie idee.\\Povero cuore,\\io ti metto alla prova ma povero me.}

%%%%%%%%%%%%%%%%%%%%
% Inizio documento %
%%%%%%%%%%%%%%%%%%%%

\begin{document}
\frontmatter

\maketitle

\chapter*{Ringraziamenti/Acknowledgements}
\thispagestyle{empty}
dedica a parenti e amici

The research work for this thesis was carried out at the Laboratory for Automation and Robotics (LAR) in the Department of Mechanical Engineering of Aveiro (Portugal), during a period of 5 months as an exchange student.
\begin{figure}[!h]
	\centering
	\includegraphics[width=0.50\textwidth]{../figure/logo_ua.png}
	\label{fig:logo_ua}
\end{figure}

I would like to express my gratitude to Professor Vitor Santos, for welcoming me at the Laboratory for Automation and Robotics and making possible my experience of study at the University of Aveiro. His help has been invaluable both personally and academically. Moreover I would like to thank Professor Angelo Cenedese for his advices and his help during this work.



\clearemptydoublepage

\chapter*{Abstract}
\thispagestyle{empty}
abstact

\clearemptydoublepage

% Indice della tesi
\setcounter{page}{1}
\tableofcontents

\clearemptydoublepage

% Lista delle figure
\listoffigures
\clearemptydoublepage


\mainmatter

\chapter{Introduction}
%\addcontentsline{toc}{chapter}{Introduction}
\section{ATLAS Project}
\section{Context of the Problem and Motivation}
\section{Proposed Approach}
\section{Thesis Outline and Contributions}

\clearemptydoublepage

\chapter{Literature Review}
In this chapter we introduce the literature survey of the various algorithms used for robot navigation. The navigation methods are divided into two kinds of control: global path planning and local motion control. First we analyze these two types of strategies and then we describe the current situation of Model Prediction which is the control method that we have adopted.
\section{Global Path Searching Method}
Global path planning uses information about a priori model or a map of the enviroment to evaluate the shortest path that allows the motion from a starting point to the goal position.
There are a lot of path planning algorithms, like cell decomposition, road map or potential field, where the calculation of a complete trajectory from a start position to one or more goal positions can be computed off-line. However they produce a reliable path if and only if the map of the enviroment is already available. So, this prior knowledge of the space where the robot can move, must be accessible.
The most famous method developed for global navigation is the Dijkstra algorithm \cite{Dijkstra:1959}. In recent years, an evolution of this method called A* is one of the most used \cite{Hart:1968}: this algorithm gives a complete and optimal global path in static enviroments. However, it was upgraded in D* \cite{Stentz93optimaland} for efficient online searching of a dynamic enviroment, which gives sequences of path points in the known or partially known space. The basic idea of these two methods is the following: minimizing their cost functions, these strategies have the capability of a quickly  redesign when the condistions of the space change, to guarantee an optimal solution of the trajectory from the starting point to the goal location.  
\subsection{The A* Algorithm}
\subsection{The D* Algorithm}
\section{Local Motion Control}
Local Motion Control is related to the real-time motion of the robot inside in unknown enviroment, where monitoring with the sensors, it can detect where are the obstacles and generate a motion to avoid the collision with them.
\subsection{Potential Field Method}
\subsection{Dynamic Window Approach}
\subsection{VFF Approach for Obstacle Avoidance}
\section{MPC in Autonomous Driving}
\cite{4651075}
\cite{vsantos2010}
\cite{borelli}
\cite{Skoda:Thesis:2016}

\clearemptydoublepage

\chapter{Model Predictive Control}

In this chapter, the theory of Model Predictive Control is discussed in detail to highlight working principle. In particular for this work we used an advanced control strategy based on this paradigm called Adaptive MPC that uses a fixed model structure, but allows the model parameters to evolve with the time.

\section{Generic Model Predictive Control problem}
Model Predictive Control (MPC), also known as Moving Horizon Control (MHC) or Receding Horizon Control (RHC), is a popular method for the control of slow dynamical systems, to generate the required
control inputs that are calculated at each sampling instance $k$, using the current state as initial
conditions to solve a finite optimal control problem. Some of the advantages of using MPC are:
\begin{itemize}
\item the ability to handle unstable, time variable, non-minimum phase systems;
\item robustness feature with the uncertainties in the nonlinear systems;
\item built in feed-forward control to handle disturbances in the processes;
\item enhanced tuning features to achieve the best response including transient responses;
\item the possibility to introduce constraints in a natural form;
\item if the references are known in advance, they can be used in order to optimize the reference tracking.
\end{itemize}


The methodology of all the controllers belonging to the MPC family is characterized by the following strategy, represented in Figure \ref{fig:mpc_theory}. The future outputs for a  determined horizon, called  the  prediction horizon, are predicted at each instant $k$ using the process model. These predicted outputs depend on the known values up to instant $k$ (past inputs and outputs) and on the future control signals which are those to be sent to the system and calculated. The set of future control signals is calculated by optimizing a determined criterion to keep the process as close as possible to the reference trajectory. This criterion usually takes the form of a quadratic function of the errors between the predicted output signal and the predicted reference trajectory. The control effort is included in the objective function in most
cases. An explicit solution can be obtained if the criterion is quadratic, the model is linear, and there are no constraints; otherwise an iterative
optimization method has to be used. Some assumptions about the structure of the future control law are also made in some cases, such as that it will be constant from a given instant. Only the current control signal is send to the process. At the next sampling instant the measured output is evaluated and the sequence is repeated and all the steps brought up to date. Thus the predicted control input is then calculated using the receding horizon concept.
\begin{figure}[!h]
	\centering
	\includegraphics[width=\textwidth]{../figure/mpc_theory.png}
	\caption{A discrete Model Predictive Control scheme adapted from \cite{mpctoolbox}.}
	\label{fig:mpc_theory}
\end{figure}

MPC is typically formulated in the state space. For a given discrete linear time-invariant (LTI) system:
\begin{equation}
\label{eqn:MPC_plant_discrete}
\vec{x}(k+1)=\vec{A}\vec{x}(k)+ \vec{B} \vec{u}(k)
\end{equation}
where $\vec{x}(k)\in\mathbb{R}^n$, $\vec{u}(k)\in\mathbb{R}^m$ are the state and the input, respectively. The central idea in the Model Predictive Control is to minimize some cost function, while still ensuring that some constraints are fulfilled. The generic MPC problem can be written as follows:
\begin{equation}
\label{eqn:MPC_optimization}
\begin{aligned}
& \underset{\textbf{u}}{\text{minimize}}
& & J(\vec{x}(k), \textbf{u}) \\
& \text{subject to}
& & \vec{x}_{k+i+1} = \vec{A}\vec{x}_{k+i}+ \vec{B} \vec{u}_{k+i}\quad\forall i=0,\dots N-1;\\
& & & \vec{x}_{k+i}\in \mathbb{X}\quad\forall i=0,\dots N-1;\\
& & & \vec{u}_{k+i}\in \mathbb{U}\quad\forall i=0,\dots N-1;\\  
& & & \vec{x}_{k+N}\in \mathbb{X}_f;\quad\vec{x}_k = \vec{x}(k).
\end{aligned}
\end{equation}
where $\textbf{u}=(\vec{u}_k,\dots,\vec{u}_{k+N-1})$ is a sequence of control inputs, $\vec{x}_{k+i}$ is the state at time $k+i$ as predicted at time $k$, and $N$ is the prediction horizon. The sets $\mathbb{X}\in\mathbb{R}^n$ and $\mathbb{U}\in\mathbb{R}^m$ define the constraints on the state and the input, respectively. Finally, the set $\mathbb{X}_f\subseteq\mathbb{X}$ defines the terminal constraint on the state. If we consider a regulation problem, the system (\ref{eqn:MPC_plant_discrete}) should be steered to the origin and the cost function $J(\vec{x}(k), \textbf{u})$ could be in a quadratic form as follows:
\begin{equation}
\label{eqn:MPC_cost_function_regulation}
	J(\vec{x}(k), \textbf{u}) = \vec{x}^\intercal_{k+N}\vec{P}_f\vec{x}_{k+N}+\sum_{i=1}^{N}\Big(\vec{x}^\intercal_{k+i}\vec{Q}\vec{x}_{k+i}+\vec{u}_{k+i}^\intercal\vec{R}\vec{u}_{k+i}\Big)
\end{equation}
where $\vec{P}_f,\vec{Q}\geq0$ (positive semi-definite) and $\vec{R}>0$ (positive definite) are weighting matrices.

Instead if we consider a servo problem, like tracking of a reference signal, the cost function is changed as follows:
\begin{equation}
\label{eqn:MPC_cost_function_servo}
\begin{aligned}
J(\vec{x}(k), \textbf{u})=& (\vec{x}_{k+N}-\vec{x}^\text{ref}_{k+N})^\intercal\vec{P}_f(\vec{x}_{k+N}-\vec{x}^\text{ref}_{k+N})\\
&+\sum_{i=1}^{N}\Big((\vec{x}_{k+i}-\vec{x}^\text{ref}_{k+i})^\intercal\vec{Q}(\vec{x}_{k+i}-\vec{x}^\text{ref}_{k+i})+\vec{u}_{k+i}^\intercal\vec{R}\vec{u}_{k+i}\Big)
\end{aligned}
\end{equation}
where $\vec{x}^\text{ref}_{k+i}$, $\vec{x}^\text{ref}_{k+N}$ describe the reference trajectory. The standard MPC algorithm can be summarized by the following steps:
\begin{algorithm}%[b]
	\caption{Basic Model Predictive Control loop}
	\small
	\begin{algorithmic}[1]
		\State Measure the current state $\vec{x}(k)$;
		\State Solve the optimization problem \ref{eqn:MPC_optimization} with $\vec{x}(k)$ as initial state, where $\vec{u}(k)$ is calculated;
		\State Apply the first control of the optimal control sequence;
		\State Wait one sampling time and repeat steps 1-3;
	\end{algorithmic}
	\label{alg:MPCloop}
\end{algorithm}

An MPC has many strengths. Given that the model is discrete and linear it handles multivariable problems very well. Also mathematical convexity is an important part of the resulting problem formulation of an MPC. In fact there exists efficient solvers for convex optimization problems but it is therefore desirable that the MPC problem \ref{eqn:MPC_optimization} is convex which is ensured if:
\begin{enumerate}
\item the cost function is convex;
\item the prediction model is linear;
\item the constraint sets $\mathbb{X}, \mathbb{U}$ are convex.	
\end{enumerate}	
The optimization handles actuator constraints and state constraints naturally in the optimization which allows for
the process to be operated much closer to the hard constraints, which improves control performance and efficiency. Because of its predictive nature it is able to solve a variety of problems and handle disturbances smoothly.

\subsection{Tuning Parameters}
The two most important parameters to tune in order to satisfy the control objectives are the diagonal matrices $\vec{Q}$ and $\vec{R}$ that can be used to weight the system state matrix and the control inputs respectively. The response of the system that is too slow can be influenced by adding high weighting values in the $\vec{Q}$ matrix, whereas the control gains are damped with high weighing values in the $\vec{R}$ matrix. Find an optimal trade-off is a fundamental aspect for the constroller behaviour.

\subsection{Stability of MPC controller}
A limited horizon on the MPC problem affects the stability of the controllers; in order to avoid this problem it is possible to set an infinite horizon, impose end point constraints, terminal cost function or use other techniques. To obtain a stable controller, the parameters to tune are: the terminal cost, prediction horizon and constraints. Also the weights on the cost function can be tuned to ensure a stabilizing solution.

\subsection{Robustness}
If the stability can be guaranteed and the performance specifications are met with respect to a certain set of uncertainties, the system is said to be robust; in particular a controller with this property has to ensure that the constraints are never violated for any admissible disturbance realization. The uncertainties in a system are due to external disturbances, measurement noise, inaccurate values of the model parameters, non-linearities etc...
The most common type of uncertainties considered in the literature is additive disturbance because usually the current state of the system can be measured hence there is no noise in the measurements.

\section{Adaptive Model Predictive Control}
We understood that Model Predictive Control is an advanced method that predicts future behavior using a linear-time-invariant (LTI) dynamic model. These predictions are not exact and a good strategy is to make MPC insensitive to prediction errors. If the plant is strongly nonlinear or its characteristics vary dramatically with time, MPC performance might become unacceptable because LTI prediction accuracy degrade \cite{mpctoolbox}. A method that can address this degradation by adapting the prediction model for changing operating conditions is called Adaptive MPC: this control strategy uses a fixed model structure, but allows the model parameters to evolve with time. Ideally, whenever the controller requires a prediction, it uses a model appropriate for the current conditions. At each control interval, the adaptive MPC controller updates the plant model and nominal conditions. Once updated, the model and conditions remain constant over the prediction horizon. The plant model used as the basis for the adaptive MPC must be an LTI discrete-time, state-space model with a structure as follows:
\begin{equation}
\label{eqn:Adaptive_MPC_plant_discrete}
\begin{aligned}
\vec{x}(k+1)&=\vec{A}\vec{x}(k)+ \vec{B}_u \vec{u}(k)+\vec{B}_v \vec{v}(k)+\vec{B}_d \vec{d}(k)\\
\vec{y}(k)&=\vec{C}\vec{x}(k) + \vec{D}_v \vec{v}(k)+ \vec{D}_d \vec{d}(k)
\end{aligned}
\end{equation}
where the matrices \vec{A}, $\vec{B}_u$, $\vec{B}_v$, $\vec{B}_d$, \vec{C}, $\vec{D}_v$ and $\vec{D}_d$ can vary with time. The other parameters in the previous expression (\ref{eqn:Adaptive_MPC_plant_discrete}) are:
\begin{itemize}
	\item $k$ is the time index/current control interval;
	\item \vec{x} are the plant model states;
	\item \vec{u} are the manipulated inputs that can be adjusted by the MPC controller;
	\item \vec{v} are the measured disturbance inputs;
	\item \vec{d} are the unmeasured disturbance inputs;
	\item \vec{y} are the plant outputs, including both measured (necessary at least one) and unmeasured.
\end{itemize}
In the adaptive MPC control, there are additional requirements for the plant model, like the sample time $T_s$ that has to be constant and identical to the MPC control interval. This control strategy prohibits direct feed-through from any manipulated variable to any plant output. Thus, $\vec{D_v} = \vec{0}$ in the above model.
A traditional MPC controller includes a nominal operating point at which the plant model applies, such as the condition at which you linearize a nonlinear model to obtain the LTI approximation (equilibrium, reference trajectory and the	
most updated value) \cite{mpctoolbox}. In adaptive MPC, as time evolves it should update the nominal operating point to be consistent with the updated plant model. It is possible to rewrite the plant model in terms of deviations from the nominal conditions as follows:
\begin{equation}
\label{eqn:Adaptive_MPC_nominal_condition}
\begin{aligned}
\vec{x}(k+1)&=\overline{\vec{x}}+\vec{A}(\vec{x}(k)-\overline{\vec{x}})+ \vec{B}(\vec{u}_t(k)-\overline{\vec{u}}_t)+\overline{\Delta \vec{x}} \\
\vec{y}(k)&=\overline{\vec{y}}+\vec{C}(\vec{x}(k)-\overline{\vec{x}}) + \vec{D}(\vec{u}_t(k)-\overline{\vec{u}}_t)
\end{aligned}
\end{equation}
where the matrices \vec{A}, \vec{B}, \vec{C} and \vec{D} are updated with respect to time. The other parameters in the previous structure (\ref{eqn:Adaptive_MPC_nominal_condition}) are:
\begin{itemize}
	\item $\vec{u}_t$ is the combined plant input variable, comprising $\vec{u}$, $\vec{v}$ and $\vec{d}$ variables defined earlier;
	\item $\overline{\vec{x}}$ are the nominal states;
	\item $\overline{\Delta \vec{x}}$ are the nominal state increments;
	\item $\overline{\vec{u}_t}$ and $\overline{\vec{y}}$ are the nominal inputs and outputs.
\end{itemize} 
The adaptive MPC uses a Kalman filter to update its controller states which include the plant, the disturbance and measurement noise model states. In particular this filter is linear-time-varying (LTV) because adjusts the gains at each control
interval to maintain consistency with the updated plant model.

\clearemptydoublepage

\chapter{Moving Obstacle Avoidance}
\section{Problem Formulation}
\section{Design of Adaptive MPC}
\section{Simulation Results}

\clearemptydoublepage

\chapter{Lane Following}
\section{Problem Formulation}
A lane-following system is a control system that keeps the vehicle traveling along the centerline of a highway lane, while maintaining a user-set velocity. Figure \ref{fig:laneFollowing} illustrates a typical lane following scenario.
\begin{figure}[!h]
	\centering
	\includegraphics[width=0.75\textwidth]{../figure/laneFollowing/laneFollowing.pdf}
	\caption{Problem description of a lane following system.}
	\label{fig:laneFollowing}
\end{figure}

In a classic lane keeping assist, it is assumed that the longitudinal velocity is constant \cite{Adaptive_Mpc_Lane_keeping_borelli}. This restriction is relaxed in this model because the longitudinal acceleration varies in this MIMO control system. This lane-following system manipulates both the longitudinal acceleration and the front steering angle of the vehicle to keep the lateral deviation and the relative yaw angle small and the longitudinal velocity close to a driver set velocity. If these two goals cannot be met at the same moment, the system tries to balance them. The model that we are considering contains many parameters. The first fundamental block describes the vehicle dynamics: we have applied the bicycle model of lateral vehicle dynamics and approximate the longitudinal dynamics using a time constant obtaining  a linear model.
\subsection{Longitudinal Dynamics}
We can use the following state space to describe the longitudinal model:
\begin{equation}
\label{eqn:longi_dynamics_simple_model_ss}
\begin{array}{ll}
\dot{\vec{x}}_{\text{lon}} =\vec{A}_m \vec{x}_{\text{lon}}+ \vec{B_m}\vec{u}_{\text{lon}}\\
\vec{y}_{\text{lon}} =\vec{C_m} \vec{x}_{\text{lon}} + \vec{D_m} \vec{u}_{\text{lon}}
\end{array}
\end{equation}
where the input is the acceleration and the states are the longitudinal velocity and the actual acceleration which is also the only output of this system.
\begin{equation}
\vec{x}_{\text{lon}} = \begin{bmatrix}
\dot{V}_x\\V_x
\end{bmatrix},
\qquad
\vec{u}_{\text{lon}} = a
\end{equation}
and
\begin{equation}
\begin{array}{cc}
\vec{A_m}=\begin{bmatrix}
-\frac{1}{\tau}&0\\1&0
\end{bmatrix},
\qquad
\vec{B_m}=\begin{bmatrix}
\frac{1}{\tau}\\
0
\end{bmatrix},\\\\
\vec{C_m}=\begin{bmatrix}
1&0
\end{bmatrix}, 
\qquad
\vec{D_m}=0.
\end{array}
\end{equation}
where $\tau$ is a time constant \cite{long_tf}; in practice we are considering a second order transfer function like in \cite{longitudinal}.
\subsection{Lateral Dynamics}
Local function: we have a continuous vehicle lateral model from parameters obtained by simplifying the one in \cite{rathai}: 
\begin{equation}
\label{eqn:lateral_dynamics_simple_model}
\begin{array}{ll}
\dot{\vec{x}}_{\text{lat}} =\vec{A}_g \vec{x}_{\text{lat}}+ \vec{B}_g \vec{u}_{\text{lat}}\\
\vec{y}_{\text{lat}} =\vec{C}_g \vec{x}_{\text{lat}} + \vec{D}_g \vec{u}_{\text{lat}}
\end{array}
\end{equation}
where the input is the steering angle in radians, and the outputs are the lateral velocity in meters per second and yaw angle rate in radians per second:
\begin{equation}
\vec{x}_{\text{lat}} = \begin{bmatrix}
V_y\\\dot{\psi}
\end{bmatrix}
\qquad
\vec{u}_{\text{lat}} = \delta
\end{equation}
and
\begin{equation}
\begin{array}{cc}
\vec{A}_g=
\begin{bmatrix}
\displaystyle -\frac{2C_F+2C_R}{mV_x}&\displaystyle -\frac{2C_Fl_F-2C_Rl_R}{mV_x} - V_x\\
\displaystyle -\frac{2C_Fl_F-2C_Rl_R}{I_ZV_x}&\displaystyle -\frac{2C_Fl_F^2+2C_Rl_R^2}{I_ZV_x}
\end{bmatrix},
\\\\
\vec{B}_g=\begin{bmatrix}
2C_F/m\\2C_Fl_F/I_Z
\end{bmatrix},
\qquad
\vec{C}_g=\begin{bmatrix}
1&0\\0&1
\end{bmatrix}=
\vec{I}_2, 
\qquad
\vec{D}_g=\begin{bmatrix}
0\\0
\end{bmatrix}=
\vec{0}_{2\times1}.
\end{array}
\end{equation}
The parameters in the previous matrices are:
\begin{itemize}
	\item $V_x$ is the longitudinal velocity of the car;	
	\item $m$ is the total mass parameter; 
	\item $I_Z$ is the yaw moment of inertia parameter;
	\item $l_F$ and $l_R$ are the longitudinal distances from center of gravity to front and rear tires parameters;
	\item $C_F$ and $C_R$ are the cornering stiffnesses of front and rear tires parameters.
\end{itemize}
\subsection{Augmented Model for Lateral Dynamics}
The goal for the driver steering model is to keep the vehicle in its lane and follow the curved road by controlling the front steering angle . This goal is achieved by driving the yaw angle error $e_2 = \psi -\psi_{\text{des}}$ and lateral displacement error $e_1$ to zero ($\dot{e}_1 = V_xe_2+V_y$). We can incorporate these two paramenters in the augmented model:
\begin{equation}
\label{eqn:lateral_dynamics_augmented_model}
\begin{array}{ll}
\dot{\vec{x}}_{\text{aug}} =\vec{A}_a \vec{x}_{\text{aug}}+ \vec{B}_a \vec{u}_{\text{aug}}\\
\vec{y}_{\text{aug}} = \vec{C}_a \vec{x}_{\text{aug}} + \vec{D}_a \vec{u}_{\text{aug}}
\end{array}
\end{equation}
where
\begin{equation}
\vec{x}_{\text{aug}} = \begin{bmatrix}
V_y\\\dot{\psi}\\e_1\\e_2
\end{bmatrix},
\qquad
\vec{u}_{\text{aug}} = 
\begin{bmatrix}
\delta\\\dot{\psi}_{\text{des}}
\end{bmatrix}
\end{equation}
and
\begin{equation}
\begin{array}{cc} 
\vec{A}_a=\begin{bmatrix}
\vec{A}_g&\vec{0}_{2\times2}\\
\vec{I}_2&\begin{matrix}
0&V_x\\
0&0
\end{matrix}
\end{bmatrix},
\qquad
\vec{B}_a=\begin{bmatrix}
\vec{B}_g&\vec{0}_{2\times1}\\
0&0\\
0&-1
\end{bmatrix},
\\\\
\vec{C}_a=\begin{bmatrix}
\vec{0}_{2\times2}&\vec{I}_2
\end{bmatrix}, 
\qquad
\vec{D}_a=
\vec{0}_{2\times2}. 
\end{array}
\end{equation}

\subsection{Overall Model Dynamics}
Combining (\ref{eqn:longi_dynamics_simple_model_ss}) with (\ref{eqn:lateral_dynamics_augmented_model}) yields the state-space model that characterizes the Model Predictive Controller:
\begin{equation}
\label{eqn:full_dynamics_model}
\begin{array}{ll}
\dot{\vec{x}}_{\text{tot}} = \vec{A}_f \vec{x}_{\text{tot}}+ \vec{B}_f \vec{u}_{\text{tot}}\\
\vec{y}_{\text{tot}} = \vec{C}_f \vec{x}_{\text{tot}} + \vec{D}_f \vec{u}_{\text{tot}}
\end{array}
\end{equation}
where
\begin{equation}
\vec{x}_{\text{tot}} = \begin{bmatrix}
\vec{x}_{\text{lon}}\\\vec{x}_{\text{aug}}
\end{bmatrix} = \begin{bmatrix}
\dot{V}_x\\V_x\\V_y\\\dot{\psi}\\e_1\\e_2
\end{bmatrix},
\qquad
\vec{u}_{\text{tot}} = \begin{bmatrix}
\vec{u}_{\text{lon}}\\\vec{u}_{\text{aug}}
\end{bmatrix}  =
\begin{bmatrix}
a\\\delta\\\dot{\psi}_{\text{des}}
\end{bmatrix}
\end{equation}
and
\begin{equation}
\begin{array}{cc}
\vec{A}_f=\begin{bmatrix}
\vec{A}_m&\vec{0}_{2\times4}\\
\vec{0}_{4\times2}&\vec{A}_a
\end{bmatrix},
\qquad
\vec{B}_f=\begin{bmatrix}
\vec{B}_m&\vec{0}_{2\times2}\\
\vec{0}_{4\times1}&\vec{B}_a
\end{bmatrix},
\\\\
\vec{C}_f=\begin{bmatrix}
\vec{C}_m&\vec{0}_{1\times4}\\
\vec{0}_{2\times2}&\vec{C}_a
\end{bmatrix}, 
\qquad
\vec{D}_f=\vec{0}_{3\times3}. 
\end{array}
\end{equation} 

However the system to be controlled is usually modeled by a linear discrete state-space model:
\begin{equation}
\label{eqn:full_dynamics_model_disc}
\begin{array}{rr}
{\vec{x}}_{\text{tot}}(k+1) =\vec{A} \vec{x}_{\text{tot}}(k)+ \vec{B} \vec{u}_{\text{tot}}(k)\\
\vec{y}_{\text{tot}}(k) = \vec{C}\vec{x}_{\text{tot}}(k) + \vec{D} \vec{u}_{\text{tot}}(k)
\end{array}
\end{equation}
where $\vec{A}$ and $\vec{B}$ are the state and control matrices for the discrete state-space equation, respectively, which can be calculated, also in this case, with the Euler method as:
\[
\vec{A} = e^{\vec{A}_fT_s},\qquad \vec{B} = \int_{kT_s}^{(k+1)T_s} e^{\vec{A}_f[(k+1)T_s-\eta]}\vec{B}_f d\eta
\]
where $T_s$ is the sampling interval for the discrete state-space model. The matrices $\vec{C}$ and $\vec{D}$ are equivalent to those in the continuous case.

\section{Design of Adaptive Model Predictive Control}
We created an Adaptive MPC controller with a prediction model that has six states, three outputs (longitudinal velocity, lateral deviation, relative yaw angle), and two manipulated signals (acceleration and steering). 
The objective of the trajectory planning along specified path can be described as follows: given a path which the vehicle is expected to follow design a trajectory of a car-vehicle configuration.
In order to do this, according with \cite{curvature} it is possible to derive the road curvature and its derivative.
The product of the road curvature and the longitudinal velocity is modeled as a measured disturbance. We have set the constraints for manipulated variables and the scale factors. Moreover we have specified the weights in the standard MPC cost function. The third output, yaw angle, is allowed to float because there are only two manipulated variables to make it a square system. In this controller, there is no steady-state error in the yaw angle as long as the second output, lateral deviation, reaches 0 at steady state. Finally we have also penalized acceleration change more for smooth driving experience. This controller uses a linear model for the vehicle dynamics and updates the model online as the longitudinal velocity varies.

\section{Simulation Results}
The proposed adaptive MPC algorithm is designed in the MATLAB/Simulink and validated through different simulations. The objective of these test is to evaluate the behavior of the proposed control strategy in critical situations.
Table 3 shows the parameters used in the lane following simulations.
\begin{table}[!h]
	\centering
	\begin{tabular}{|c|c|}
		\hline
		Parameters          & Values      \\
		\hline
		$m$          & \SI{1575}{kg}              \\
		$I_z$         & \SI{2875}{kgm^2}               \\
		$l_F$           & \SI{1.2}{m}               \\
		$l_R$         & \SI{1.6}{m}               \\
		$C_F$          & \SI{19000}{\newton/rad}      \\
		$C_R$           & \SI{33000}{\newton/rad}  \\
		$\tau$             & 0.2                \\
		$V_0$           & \SI{15}{m/s}       \\
		$V_{\text{set}}$          & \SI{20}{m/s}   \\
		$T_s$         & \SI{0.02}{s}          \\
		\hline
	\end{tabular}
\end{table}

\subsection{Sinusoidal Path}

Figures \ref{fig:reference_laneFollowing} and \ref{fig:curvature_laneFollowing} show the desired path that the car must follow and its curvature, where the former is described in terms of the lateral position $Y_{\text{ref}}$ as function of the longitudinal position $X_{\text{ref}}$ and the latter is derived according with \cite{curvature}. The ATLASCAR2 is controlled to follow a sinusoidal trajectory which is given as follows:
\begin{equation}
X_\text{ref}=V_x\cdot t, \quad Y_\text{ref}=5\sin(X_\text{ref}/20)\quad \text{with}\quad t\in[0,20]\SI{}{s}
\end{equation}
%LANE FOLLOWING SINUSOIDAL PATH - PATH/CURVATURE
\begin{figure}[!t]
	\centering
	\begin{minipage}[t]{0.49\textwidth}
		\includegraphics[width=\textwidth]{../../MATLAB/lane_following/figure/Reference.pdf}
		\subcaption{}
		\label{fig:reference_laneFollowing}
	\end{minipage}
	\begin{minipage}[t]{0.49\textwidth}
		\includegraphics[width=\textwidth]{../../MATLAB/lane_following/figure/Curvature.pdf}
		\subcaption{}
		\label{fig:curvature_laneFollowing}
	\end{minipage}
	\caption{Desired path and curvature of the ATLASCAR2 in a simulation of \SI{20}{s}.}
	\label{fig:laneFollowing_desired}
\end{figure}

Moreover the following figures show the trend of the main parameters confirming that the control strategy used allows the vehicle to follow the path. In particular we simulated also a small error in the sensor dynamics in order to make the simulation more realistic: we added a 3 percent error to the longitudinal velocity and this is evident from the small noise in the graphs of the steering angle (Figure \ref{fig:steering_laneFollowing}) and the lateral deviation (Figure \ref{fig:lateral_deviation_laneFollowing}).
Figure {\ref{fig:longitudinal_velocity_laneFollowing}} shows the evolution of the vehicle longitudinal velocity. At the start of the simulation, this velocity is equal to the initial condition for longitudinal velocity parameter $V_0$. At run time, we can note that $V_x$ reaches the predefined value of \SI{20}{m/s} meters for second and then it stabilizes near the cruising speed because it continues to vary the steering angle to adapt to the path to be followed. Finally we presents the overall scheme for the lane following developted in Simulink depicted in Figure \ref{fig:scheme_lane_following}.
%LANE FOLLOWING SINUSOIDAL PATH - SIGNALS
\begin{figure}[!h]
	\centering
	\begin{minipage}[t]{0.49\textwidth}
		\includegraphics[width=\textwidth]{../../MATLAB/lane_following/figure/LongitudinalVelocityVsTime.pdf}
		\subcaption{Longitudinal velocity $V_x$ w.r.t. time.}
		\label{fig:longitudinal_velocity_laneFollowing}
	\end{minipage}
	\begin{minipage}[t]{0.49\textwidth}
		\includegraphics[width=\textwidth]{../../MATLAB/lane_following/figure/AccelerationVsTime.pdf}
		\subcaption{Acceleration $u_1$ w.r.t. time.}
		\label{fig:acceleration_laneFollowing}
	\end{minipage}
	\begin{minipage}[t]{0.49\textwidth}
		\includegraphics[width=\textwidth]{../../MATLAB/lane_following/figure/SteeringAngleVsTime.pdf}
		\subcaption{Steering angle $u_2$ w.r.t. time.}
		\label{fig:steering_laneFollowing}
	\end{minipage}
	%\begin{minipage}[t]{\columnwidth}
	%	\includegraphics[width=\columnwidth]{../../MATLAB/lane_following/figure/RelativeYawAngleVsTime.pdf}
	%	\subcaption{Relative yaw angle $e_2$ w.r.t. time.}
	%	\label{fig:relative_yaw_angle_laneFollowing}
	%\end{minipage}
	\begin{minipage}[t]{0.49\textwidth}
		\includegraphics[width=\textwidth]{../../MATLAB/lane_following/figure/LateralDeviationVsTime.pdf}
		\subcaption{Lateral deviation $e_1$ w.r.t. time.}
		\label{fig:lateral_deviation_laneFollowing}
	\end{minipage}
	\caption{Time signals of the ATLASCAR2 in the simulation with a sinusoidal path.}
	\label{fig:laneFollowing_signals}
\end{figure}

%OVERALL SCHEME LANE FOLLOWING
\begin{figure}[!h]
	\centering
	\includegraphics[width=\textwidth]{../figure/lane_following_AMPC.pdf}
	\caption{Overall procedure scheme lane following.}
	\label{fig:scheme_lane_following}
\end{figure}

%SENSOR DYNAMICS
\begin{figure}[!h]
	\centering
	\includegraphics[width=\textwidth]{../figure/lane_following_AMPC_sensor_dynamics.pdf}
	\caption{Sensor Dynamics of the overall scheme lane following.}
	\label{fig:scheme_lane_following_sensor_dynamics}
\end{figure}


\subsection{Simple Curve Path}


%LANE FOLLOWING SIMPLE CURVE - PATH/CURVATURE
\begin{figure}[!t]
	\centering
	\begin{minipage}[t]{0.49\textwidth}
		\includegraphics[width=\textwidth]{../../MATLAB/lane_following_curve/figure/Reference_curve.pdf}
		\subcaption{}
		\label{fig:reference_laneFollowing_curve}
	\end{minipage}
	\begin{minipage}[t]{0.49\textwidth}
		\includegraphics[width=\textwidth]{../../MATLAB/lane_following_curve/figure/Curvature_curve.pdf}
		\subcaption{}
		\label{fig:curvature_laneFollowing_curve}
	\end{minipage}
	\caption{Desired path and curvature of the ATLASCAR2 in a simulation of \SI{10}{s}.}
	\label{fig:laneFollowing_desired_curve}
\end{figure}

%LANE FOLLOWING SIMPLE CURVE - PATH/CURVATURE
\begin{figure}[!h]
	\centering
	\begin{minipage}[t]{0.49\textwidth}
		\includegraphics[width=\textwidth]{../../MATLAB/lane_following_curve/figure/LongitudinalVelocityVsTime_curve.pdf}
		\subcaption{Longitudinal velocity $V_x$ w.r.t. time.}
		\label{fig:longitudinal_velocity_laneFollowing_curve}
	\end{minipage}
	\begin{minipage}[t]{0.49\textwidth}
		\includegraphics[width=\textwidth]{../../MATLAB/lane_following_curve/figure/AccelerationVsTime_curve.pdf}
		\subcaption{Acceleration $u_1$ w.r.t. time.}
		\label{fig:acceleration_laneFollowing_curve}
	\end{minipage}
	\begin{minipage}[t]{0.49\textwidth}
		\includegraphics[width=\textwidth]{../../MATLAB/lane_following_curve/figure/SteeringAngleVsTime_curve.pdf}
		\subcaption{Steering angle $u_2$ w.r.t. time.}
		\label{fig:steering_laneFollowing_curve}
	\end{minipage}
	%\begin{minipage}[t]{\columnwidth}
	%	\includegraphics[width=\columnwidth]{../../MATLAB/lane_following_curve/figure/RelativeYawAngleVsTime_curve.pdf}
	%	\subcaption{Relative yaw angle $e_2$ w.r.t. time.}
	%	\label{fig:relative_yaw_angle_laneFollowing_curve}
	%\end{minipage}
	\begin{minipage}[t]{0.49\textwidth}
		\includegraphics[width=\textwidth]{../../MATLAB/lane_following_curve/figure/LateralDeviationVsTime_curve.pdf}
		\subcaption{Lateral deviation $e_1$ w.r.t. time.}
		\label{fig:lateral_deviation_laneFollowing_curve}
	\end{minipage}
	\caption{Time signals of the ATLASCAR2 in the simulation with a curve path.}
	\label{fig:laneFollowing_signals_curve}
\end{figure}





%%%CIAOOOOOOOOOOOOOOOOOOOOOOOOOOOOOOOOOOOOOOOOOO



\clearemptydoublepage

\include{./Chapter6/Chapter6}

\clearemptydoublepage

\chapter*{Conclusions and Future Work}
\addcontentsline{toc}{chapter}{Conclusions and Future Work}

%\backmatter

\clearemptydoublepage

\bibliographystyle{ieeetr}
\bibliography{biblio}

\clearemptydoublepage


% Ridefiniamo l'etichetta per le figure e le tabelle
\renewcommand{\figurename}{Fig.}
\renewcommand{\tablename}{Tab.}
% Ridefiniamo percentuali per inserimento figure nel testo
\renewcommand{\topfraction}{0.85}
\renewcommand{\textfraction}{0.1}
\renewcommand{\floatpagefraction}{0.75}

\end{document}